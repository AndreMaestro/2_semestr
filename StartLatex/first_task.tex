\documentclass{article}

\usepackage[T2A]{fontenc}
\usepackage[utf8]{inputenc}
\usepackage[english,russian]{babel}

\begin{document}

Даниил Хармс. 



65. СЕМЬ  КОШЕК

Вот так история! Не знаю, что делать. Я совершенно запутался. Ничего разобрать
не могу. Посудите сами: поступил я сторожем на кошачью выставку.

Выдали мне кожаные перчатки, чтобы кошки меня за пальцы не цапали, и велели
кошек по клеткам рассаживать и на каждой клетке надписывать - как которую
кошку зовут.

- Хорошо, - говорю я, - а только как зовут этих кошек?

- А вот, - говорят, - кошку,которая сле- ва, зовут Машка, рядом с ней сидит
   Пронька, потом Бубенчик, а эта Чурка, а эта Мурка, а эта Бурка, а эта
   Штукатурка.

Вот остался я один с кошками и думаю: "Выкурюка я сначала трубочку, а уж потом
рассажу этих кошек по клеткам".

Вот курю я трубочку и на кошек смотрю.

Одна лапкой мордочку моет, другая на по- толок смотрит, третья по комнате
гуляет, че- твертая кричит страшным голосом, еще две кошки друг на друга шипят,
а одна подошла ко мне и меня за ногу укусила.

Я вскочил, даже трубку уронил. 
- Вот, - кричу, - противная кошка! Ты даже и на кошку не похожа. Пронька ты или
  Чурка, или, может быть, ты Штукатурка?

Тут вдруг я понял, что я всех кошек перепутал. Которую как зовут - совершенно
не знаю.

- Эй, - кричу, - Машка! Пронька! Бубенчик! Чурка! Мурка! Бурка! Штукатурка!

А кошки на меня ни малейшего внимания не обращают.

Я им крикнул:
- Кис-кис-кис!

Тут все кошки зараз ко мне свои головы повернули.

Что тут делать?

Вот кошки забрались на подоконник, по- вернулись ко мне спиной и давай в окно
смотреть.

Вот они все тут сидят, а которая тут Штукатурка и которая тут Бубенчик?

Ничего я разобрать не могу.

Я думаю так, что только очень умный человек сумеет отгадать,как какую кошку
зовут.

66. ХРАБРЫЙ  ЕЖ

Стоял на столе ящик.

Подошли звери к ящику, стали его осматривать, обнюхивать и облизывать.

А ящик-то вдруг - раз, два, три - и открылся.

А из ящика-то - раз, два, три - змея выскочила.

Испугались звери и разбежались.

Один еж не испугался, кинулся на змею и
- раз, два, три - загрыз ее.

А потом сел на ящик и закричал: "Кукаре- ку!".

Нет, не так! Еж закричал: "Ав-авав!".

Нет, и не так! Еж закричал: "Мяу-мяу- мяу!".

Нет, опять не так! Я и сам не знаю как.

Кто знает, как ежи кричат?

67. КАРЬЕРА
         ИВАНА ЯКОВЛЕВИЧА АНТОНОВА
    
Это случилось еще до революции.
    
Одна купчиха зевнула, а к ней в рот залетела кукушка.
    
Купец прибежал на зов своей супруги и, моментально сообразив, в чем дело,
поступил самым остроумным способом.
    
С тех пор он стал известен всему населению города и его выбрали в сенат.
    
Но прослужив года четыре в сенате, несчастный купец однажды вечером зевнул, и
ему в рот залетела кукушка.
    
На зов своего мужа прибежала купчиха и поступила самым остроумным способом.
    
Слава о ее находчивости распространилась по всей губернии, и купчиху повезли в
столицу показать метрополиту.
    
Выслушиваяя длинный рассказ купчихи метрополит зевнул, и ему в рот залетела
кукушка.
    
На громкий зов метрополита прибежал Иван Яковлевич Григорьев и поступил самым
остроумным способом.
    
За это Ивана Яковлевича Григорьва переименовали в Ивана Яковлевича Антонова и
представили царю.
    
И вот теперь становится ясным, каким образом Иван Яковлевич Антонов сделал
себе карьеру.

                         8 января 1935 года.

68.
Все люди любят деньги: и гладят их, и целуют, и к сердцу прижимают, и
заворачивают их в красные тряпочки, и няньчат их, как куклу. А некоторые
заключают деньзнак в рамку, вешают его на стену и поклоняяются ему как иконе.
    
Некоторые кормят свои деньги: открывают им рты и суют туда самые жирные куски
своей пищи.
    
В жару несут деньги в холодный погреб, а зимой, в лютые морозы, бросают деньги в
печку, в огонь.
    
Некоторые просто разговаривают со своими деньгами, или читают им вслух
интересные книги, или поют им приятные песни.
    
Я же не отдаю деньгам особого внимания и просто ношу из в кошельке или в
бумажнике и по мере надобности трачу их. Шибейя!

69.
    
- Видите-ли, - сказал он, - я видел как вы с ними катались третьего дня на
  лодке. Один из них сидел на руле, двое гребли, а четвертый сидел рядом с вами
  и говорил. Я долго стоял на берегу и смотрел, как гребли те двое. Да, я могу
  смело утверждать, что они хотели утопить вас. Так гребут только перед
  убийством.
    
Дама в желтых перчатках посмотрела на Клопова.
    
- Что это значит? - сказала она. - Как это так можно особенно грести перед
  убийст- вом? И потом, какой смысл им топить меня?
    
Клопов резко повернулся к даме и сказал:

- Вы знаете, что такое медный взгяд?
    
- Нет, - сказала дама, невольно отодвигаясь от Клопова.
    
- Ага, - сказал Клопов. 
- Когда тонкая фарфоровая чашка падает со шкапа и летит вниз, то в тот момент,
  пока она еще летит по воздуху, вы уже знаете, что она коснетс пола и
  разлетится на куски. А я знаю, что если человек, взгянув на другого человека
  медным взгядом, то уж рано или поздно он не- минуемо убьет его.
    
- Они смотрели на меня медным взгядом? - спросила дама в желтых перчатках.
    
- Да, сударыня, - сказал Клопов и надел шляпу.
    
Некоторое времяя оба молчали.
    
Клопов сидел, опустив низко голову.
    
- Простите меня, - вдруг сказал он тихо.
    
Дама в желтых перчатках с удивлением смотрела на Клопова и мосчала.
    
- Это все неправда, - сказал Клопов. 

- Я выдумал про медный взгяд сейчас, вот тут, сидя с вами на скамейке. Я, видите
  ли, разбил сегодня свои часы, и мне все представляется в мрачном свете.
    
Клопов вынул из кармана платок, развер- нул его и протянул даме разбитые часы.
    
- Я носил их шестнадцать лет. Вы понимаете, что это значит? Разбить часы,
  которые шестнадцать лет тикали у меня вот тут под сердцем? У вас есть часы?
\end{document}
