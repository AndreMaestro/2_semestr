\documentclass{article}

\usepackage[T2A]{fontenc}
\usepackage[utf8]{inputenc}
\usepackage[english, russian]{babel}
\usepackage{tempora}

\begin{document}

\begin{center}
\textbf{Юрин Игорь Юрьевич}
\end{center}

\textit{Криптография} "--- это наука и практика защиты информации от несанкционированного доступа. Она обеспечивает конфиденциальность, целостность и аутентификацию данных путем преобразования открытого текста в зашифрованный формат, доступный только авторизованным сторонам

\textit{Файервол} "--- это программное или аппаратное средство, которое контролирует и фильтрует входящий и исходящий сетевой трафик на основе заранее определенных правил безопасности. Он помогает защитить компьютеры и сети от вредоносных атак и несанкционированного доступа.

\textit{Троян} "--- это тип вредоносного ПО, которое маскируется под полезную программу, но на самом деле позволяет злоумышленникам получить несанкционированный доступ к компьютеру или сети. Трояны часто используются для кражи данных или установки других вредоносных программ.

\textit{Компьютерный вирус} "--- это вредоносная программа, которая может реплицироваться и распространяться на другие компьютеры. Вирусы могут повреждать файлы, крадить данные или нарушать работу системы.

\textit{Майнинг} "--- это процесс подтверждения транзакций в блокчейне с помощью мощных вычислительных ресурсов. В обмен на подтверждение транзакций майнеры получают вознаграждение в виде криптовалюты, например, биткоинов.

\textit{Flipper zero} "--- это портативное устройство для хакеров и энтузиастов, которое позволяет исследовать и взаимодействовать с различными беспроводными протоколами, RFID, инфракрасными устройствами и другими технологиями.

\textit{Аутентификация} "--- это процесс проверки личности пользователя или системы для обеспечения доступа к определенным ресурсам или данным. Это может включать пароли, биометрические данные или другие методы идентификации.

\textit{Брутфорс} "--- это метод взлома пароля или криптографического ключа путем последовательного перебора всех возможных комбинаций. Этот метод часто используется злоумышленниками для получения несанкционированного доступа к системам или данным.

\textit{DDoS"=атака (Distributed Denial of Service)} "--- это тип кибератаки, при которой злоумышленники используют множество компьютеров для перегрузки системы или сайта трафиком, что приводит к его недоступности для легитимных пользователей.

\end{document}

