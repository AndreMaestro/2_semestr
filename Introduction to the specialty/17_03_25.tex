\documentclass{article}

\usepackage[T2A]{fontenc}
\usepackage[utf8]{inputenc}
\usepackage[english, russian]{babel}
\usepackage{tempora}

\begin{document}

\begin{center}
    \textbf{Neoflex}
\end{center}

\textit{Стек} "--- набор технологий, инструментов и программных решений, используемых для разработки и поддержки приложения или системы. Например, веб"=стек может включать HTML, CSS, JavaScript, базу данных и серверный язык программирования.

\textit{Микросервисная архитектура} "--- архитектурный стиль разработки программного обеспечения, при котором приложение разбивается на небольшие независимые сервисы, каждый из которых выполняет определённую бизнес"=функцию и взаимодействует с другими через API.

\textit{Микросервис} "--- небольшой независимый сервис в рамках микросервисной архитектуры, который отвечает за выполнение конкретной задачи или функции. Каждый микросервис разрабатывается, развёртывается и масштабируется независимо от других.

\textit{Развёртывание (Deployment)} "--- процесс переноса программного обеспечения (например, приложения или сервиса) из среды разработки в рабочую среду, где оно становится доступным для конечных пользователей.

\textit{Изоляция процесса} "--- принцип, при котором процессы или компоненты системы работают независимо друг от друга, что позволяет минимизировать влияние сбоев в одном процессе на другие. В микросервисной архитектуре это достигается за счёт независимости сервисов.
\textit{Монолитная архитектура} "--- архитектурный подход, при котором приложение разрабатывается как единое целое, где все компоненты (например, база данных, серверная и клиентская части) тесно связаны и работают в рамках одного процесса.

\textit{Деплой (Deploy)} "--- то же, что и развёртывание. Процесс вывода нового или обновлённого программного обеспечения в эксплуатацию, включая его установку, настройку и запуск в рабочей среде.

\textit{Децентрализация} "--- принцип распределения управления, данных или процессов между несколькими независимыми узлами или компонентами, что позволяет избежать единой точки отказа и повысить отказоустойчивость системы.

\textit{Agile (Гибкая методология разработки)} "--- подход к управлению проектами и разработке программного обеспечения, который предполагает итеративную и инкрементальную работу, гибкость к изменениям и тесное взаимодействие с заказчиком. Основной акцент делается на быструю адаптацию к требованиям.

\textit{Kanban (Канбан)} "--- метод управления рабочими процессами, который визуализирует задачи на доске (обычно разделённой на столбцы, например, <<Сделать>>, <<В процессе>>, <<Готово>>). Цель "--- оптимизировать поток задач и минимизировать время выполнения.

\end{document}