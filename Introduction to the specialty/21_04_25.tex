\documentclass{article}

\usepackage[T2A]{fontenc}
\usepackage[utf8]{inputenc}
\usepackage[english, russian]{babel}
\usepackage{tempora}

\begin{document}

\begin{center}
    \textbf{КБПА}
\end{center}

\textit{Функциональные требования} "--- это требования, определяющие, что должна делать система: её функции, возможности и бизнес-логику.

\textit{Нефункциональные требования} "--- это требования, определяющие, как система должна работать: производительность, безопасность, масштабируемость и т.д.

\textit{С} "--- процедурный язык, широко используется в системном программировании.

\textit{С++} "--- расширение C с поддержкой ООП, шаблонов и STL.

\textit{Accembler} "--- низкоуровневый язык программирования, близкий к машинному коду. Используется для оптимизации критичных участков кода или работы с железом.

\textit{QT creator} "--- кроссплатформенная IDE для разработки на C++ с использованием фреймворка Qt. Позволяет создавать GUI-приложения.

\textit{yaml} "--- человекочитаемый формат для сериализации данных. Часто используется для конфигурационных файлов (например, в Docker, Kubernetes).

\textit{git} "--- распределённая система контроля версий. Позволяет отслеживать изменения в коде, работать в команде и управлять ветками.

\textit{svn} "--- централизованная система контроля версий. Альтернатива Git, но с другим подходом к хранению истории изменений.

\textit{HT (Highload Technologies)} "--- технологии для работы с высокими нагрузками (например, горизонтальное масштабирование, кэширование, асинхронная обработка).

\textit{HTTP} "--- протокол передачи данных в интернете. Основа web (клиент-серверное взаимодействие).

\textit{Тестирование ПО} "--- процесс проверки соответствия программы требованиям

\textit{JS} "--- язык программирования для создания интерактивных web-страниц. Запускается в браузере или на сервере (Node.js). 

\end{document}