\documentclass{article}

\usepackage[T2A]{fontenc}
\usepackage[utf8]{inputenc}
\usepackage[english, russian]{babel}
\usepackage{tempora}

\begin{document}

\begin{center}
    \textbf{Кузнецов Александр}
\end{center}

\textit{Машинное обучение (ML)} "--- направление искусственного интеллекта, позволяющее компьютерам учиться на основе данных и совершенствовать свои результаты без прямого программирования, используя алгоритмы и статистические модели.

\textit{Нейронные сети} "--- вычислительные системы, созданные по аналогии с биологическими нейронами, состоящие из нескольких слоев искусственных нейронов. Они применяются для решения задач машинного обучения, таких как распознавание изображений и предсказание событий.

\textit{Квантовый компьютер} "--- вычислительная система, основанная на принципах квантовой механики (таких как суперпозиция и запутанность). Она способна решать сложные задачи, недоступные традиционным компьютерам.

\textit{Большие данные (Big Data)} "--- огромные массивы структурированной и неструктурированной информации, обработка которых требует особых инструментов и методик для анализа, хранения и извлечения полезных сведений.

\textit{Анализ данных} "--- процесс изучения, очистки, трансформации и моделирования данных с целью получения ценных выводов, обнаружения закономерностей и помощи в принятии обоснованных решений.

\textit{Нейроуправление} "--- подход к управлению системами, который основывается на применении нейронных сетей для обучения и адаптации к меняющимся условиям. Часто используется в робототехнике и автоматизации.

\textit{Виртуальная реальность (VR)} "--- технология, создающая полностью смоделированное компьютером окружение, воспринимаемое пользователями как настоящее благодаря специальным устройствам вроде VR-шлемов.

\textit{Дополненная реальность (AR)} "--- технология, дополняющая реальную среду цифровыми элементами (например, изображениями, текстом или анимациями), которые отображаются на экранах устройств, таких как смартфоны или AR-очки.

\textit{Перцептрон} "--- один из ранних алгоритмов машинного обучения, представляющий собой простую модель искусственного нейрона, используемую для классификации данных.

\textit{Кубит (квантовый бит)} "--- основная единица информации в квантовом компьютере, способная одновременно находиться в состоянии 0 и 1 (суперпозиция), что обеспечивает возможность параллельных вычислений.

\textit{GitLab} "--- платформа для управления разработкой ПО, объединяющая контроль версий (через Git), средства непрерывной интеграции и доставки (CI/CD), управление задачами и совместную работу.

\textit{Obsidian} — программа для организации личных знаний, позволяющая создавать и управлять заметками в формате Markdown, поддерживает создание связей между ними и визуализирует графы взаимосвязей.

\end{document}