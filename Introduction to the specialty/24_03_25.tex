\documentclass{article}

\usepackage[T2A]{fontenc}
\usepackage[utf8]{inputenc}
\usepackage[english, russian]{babel}
\usepackage{tempora}

\begin{document}

\begin{center}
    \textbf{Мастер Софт}
\end{center}

\textit{Java} "--- высокоуровневый объектно-ориентированный язык программирования, разработанный компанией Sun Microsystems (позже приобретён Oracle). Отличается кроссплатформенностью благодаря использованию виртуальной машины (JVM).  

\textit{MySQL} "--- реляционная система управления базами данных (СУБД) с открытым исходным кодом. Широко используется в веб"=разработке (например, в связке с PHP).  

\textit{1C} "--- платформа для автоматизации бизнес-процессов, бухгалтерского учёта и управления предприятием. Включает в себя конфигурируемые программные решения (например, «1С:Бухгалтерия», «1С:ERP»).  

\textit{Turbo Pascal} – среда разработки и компилятор для языка Pascal, созданный Borland в 1980=х. Был популярен для обучения программированию.  

\textit{JSP (JavaServer Pages)} "--- технология для создания динамических веб"=страниц на Java, позволяющая встраивать Java"=код в HTML.  

\textit{1C:ERP} "--- ERP"=система на платформе <<1С>>, предназначенная для комплексного управления ресурсами предприятия (финансы, логистика, производство и др.).  

\textit{ООП (Объектно-ориентированное программирование)} "--- парадигма программирования, основанная на использовании объектов, классов, инкапсуляции, наследования и полиморфизма. Примеры языков: Java, C++, Python.  

\textit{Hard Skills (<<жёсткие>> навыки)} "--- технические, измеримые профессиональные умения (например, владение языком программирования, работа с Excel).  

\textit{Soft Skills (<<мягкие>> навыки)} "--- личностные качества и социальные навыки (коммуникация, управление временем, работа в команде).  

\textit{Базы данных} – организованные структуры для хранения, управления и обработки данных. Бывают реляционные (MySQL, PostgreSQL) и NoSQL (MongoDB).  

\textit{Операционные системы} "--- программное обеспечение, управляющее ресурсами компьютера и обеспечивающее работу приложений. Примеры: Windows, Linux, macOS.  

\textit{ИИ (Искусственный интеллект)} "--- область информатики, занимающаяся созданием систем, способных выполнять задачи, требующие человеческого интеллекта (машинное обучение, нейросети).  

\textit{1С:Конфигуратор} "--- инструмент для разработки и настройки конфигураций (программных решений) в системе <<1С:Предприятие>>.  

\textit{1С:EPT (1С:Электронное обучение)} "--- система дистанционного обучения и тестирования на платформе <<1С>>, используется для подготовки сотрудников и сертификации.  

\end{document}
