\documentclass{article}

\usepackage[T2A]{fontenc}
\usepackage[utf8]{inputenc}
\usepackage[english, russian]{babel}
\usepackage{tempora}

\begin{document}

\begin{center}
    \textbf{СИБИНТЕК}
\end{center}

\textit{1C} "--- это российская компания, специализирующаяся на разработке программного обеспечения для автоматизации бизнес-процессов.

\textit{Галактика} "--- это высокопроизводительная российская комплексная информационная система управления ресурсами предприятий и холдинговых структур

\textit{Бизнес"=процесс} "--- это последовательность действий или задач, направленных на достижение определенной бизнес-цели. Это может включать различные операции, такие как производство, продажи или управление персоналом

\textit{ERP"=системы (Enterprise Resource Planning)} "--- это программные решения, предназначенные для интеграции и автоматизации всех аспектов бизнеса, включая производство, логистику, бухгалтерский учет и управление персоналом. Они обеспечивают централизованное управление данными и процессами

\textit{Бизнес"=анализ} "--- это процесс изучения и улучшения бизнес"=процессов внутри организации. Он включает в себя сбор и анализ данных для выявления проблем и разработки решений для повышения эффективности бизнеса.

\textit{MS"=Visio} "--- это программное обеспечение для создания диаграмм и схем, используемое для визуализации бизнес"=процессов, сетей и других систем.

\textit{ARIS (Architecture of Integrated Information Systems)} "--- это инструмент для моделирования и анализа бизнес"=процессов, позволяющий создавать и оптимизировать процессные модели.

\textit{PI"=System} "= это система управления данными и аналитикой, используемая для сбора и анализа данных из различных источников, особенно в промышленных и производственных средах.

\textit{Web"=Tutor} "--- это система для создания и управления онлайн"=курсами и образовательными ресурсами

\textit{Directum} "--- это система электронного документооборота и управления бизнес"=процессами, используемая для автоматизации документооборота и оптимизации рабочих процессов.

\textit{MS SharePoint} "--- это платформа для сотрудничества и управления контентом, позволяющая пользователям хранить и обмениваться документами, а также создавать рабочие пространства для команд.

\textit{КСЭД} "--- это российская система для автоматизации электронного документооборота и управления бизнес"=процессами.

\textit{ЛИМС} — это система для управления лабораторными данными и процессами, используемая в научных и производственных лабораториях.

\textit{Реляционные базы данных} — это тип баз данных, в которых данные организованы в виде таблиц, связанных между собой отношениями.

\textit{Базы данных реального времени} "--- это системы, которые обрабатывают и предоставляют данные в режиме реального времени, часто используемые в приложениях, требующих быстрого доступа к данным.

\textit{Java} "--- это объектно"=ориентированный язык программирования, широко используемый для разработки приложений, веб"=сайтов и мобильных приложений.

\textit{C\#} "--- это современный, объектно-ориентированный язык программирования, разработанный компанией Microsoft, часто используемый для создания приложений на платформе .NET.

\textit{SQL (Structured Query Language)} "--- это язык для управления и запроса данных в реляционных базах данных.

\textit{DevExpress} "--- это компания, производящая инструменты и компоненты для разработки программного обеспечения, особенно для платформы .NET.

\textit{ITSM} "--- это набор практик и процессов, направленных на управление и поддержку IT"=сервисов внутри организации.

\textit{Системный администратор} "--- это специалист, ответственный за управление и поддержку компьютерных систем, сетей и инфраструктуры.

\textit{Redis} "--- это база данных в памяти, используемая для хранения и обработки данных в режиме реального времени.

\textit{Kafka} "--- это распределенная система обмена сообщениями, используемая для обработки и передачи больших объемов данных в реальном времени

\textit{k8s (Kubernetes)} — это система для автоматизации развертывания, масштабирования и управления контейнеризированными приложениями.

\textit{Микросервисная архитектура} "--- это подход к разработке программного обеспечения, при котором приложение разбивается на небольшие, независимые сервисы, которые взаимодействуют друг с другом.

\textit{Монолитная архитектура} "--- это подход к разработке программного обеспечения, при котором все компоненты приложения построены как единое целое, без разделения на отдельные сервисы.

\end{document}