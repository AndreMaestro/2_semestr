\documentclass{article}

\usepackage[T2A]{fontenc}
\usepackage[utf8]{inputenc}
\usepackage[english, russian]{babel}
\usepackage{tempora}

\begin{document}

\begin{center}
    \textbf{БАЗАЛЬТ}
\end{center}

\textit{Microsoft SQL Server} "--- система управления реляционными базами данных (СУБД) от Microsoft. Поддерживает язык SQL, транзакции, аналитику и интеграцию с другими продуктами Microsoft.

\textit{Azure SQL Database} "--- облачная СУБД от Microsoft в рамках Azure. Предлагает масштабируемые базы данных как сервис (PaaS) с высокой доступностью и безопасностью.

\textit{Microsoft SharePoint} "--- платформа для совместной работы, управления документами и создания корпоративных порталов. Интегрируется с Office 365 и Active Directory.

\textit{Active Directory} "--- служба каталогов от Microsoft для управления пользователями, компьютерами и правами в доменной сети Windows. Основана на LDAP и Kerberos.

\textit{Debian} "--- один из старейших и наиболее стабильных дистрибутивов Linux. Основа для Ubuntu, Kali Linux и других. Известен строгим подходом к свободному ПО.
	
\textit{Red Hat} "--- компания, разрабатывающая коммерческие продукты на основе Linux (RHEL "--- Red Hat Enterprise Linux). Также владеет OpenShift, Ansible и Fedora.
	
\textit{SUSE} "--- европейский дистрибутив Linux (openSUSE "--- бесплатный, SUSE Linux Enterprise "--- коммерческий). Популярен в корпоративной среде.
	
\textit{GNOME} "--- одна из основных графических оболочек (десктоп"=сред) для Linux. Известна удобным интерфейсом и открытостью.

\textit{Make} "--- утилита для автоматизации сборки программ. Читает инструкции из Makefile и управляет компиляцией кода.
	
\textit{CMake} "--- кроссплатформенная система управления сборкой. Генерирует Makefile или файлы проектов для IDE (Visual Studio, Xcode).
	
\textit{TCP/IP} "--- стек сетевых протоколов для передачи данных в интернете. Включает TCP (надёжная передача), IP (маршрутизация), HTTP, FTP и др.

\textit{Samba} "--- реализация протокола SMB/CIFS для Linux/Unix. Позволяет совместно использовать файлы и принтеры в сети с Windows.

\textit{Citrix} "--- компания, разрабатывающая решения для виртуализации (Citrix Virtual Apps, XenDesktop) и удалённого доступа.

\textit{DNS (Domain Name System)} "--- система преобразования доменных имён (например, google.com) в IP"=адреса (например, 8.8.8.8).
	
\textit{HTTP (HyperText Transfer Protocol)} "--- протокол передачи веб-страниц. Основа интернета (HTTPS "--- защищённая версия с шифрованием).
	
\textit{DHCP (Domain Host Configuration Protocol)} "--- протокол автоматической выдачи IP"=адресов и сетевых настроек устройствам в локальной сети.

\textit{i586} "--- устаревшее обозначение процессоров архитектуры x86 (Intel Pentium и аналоги). Иногда используется в контексте совместимости ПО.
	
\textit{x86\textunderscore64} "--- 64"=битная версия архитектуры x86. Поддерживает больше памяти и более высокую производительность (используется в современных ПК).
	
\textit{aarch64} "--- 64"=битная версия архитектуры ARM. Используется в смартфонах (Android, iOS), серверах (AWS Graviton) и новых ПК (Apple M1/M2).
	
\textit{LoongArch} "--- собственная китайская процессорная архитектура (разработка Loongson). Альтернатива x86 и ARM для снижения зависимости от западных технологий.
	
\textit{RISC"=V} "--- открытая и свободная архитектура процессоров. Набирает популярность в встраиваемых системах и суперкомпьютерах.
	
\textit{e2k} "--- российская процессорная архитектура (разработка МЦСТ). Используется в компьютерах "Эльбрус" для спецприменений.

\end{document}