\documentclass{article}

\usepackage[T2A]{fontenc}
\usepackage[utf8]{inputenc}
\usepackage[english, russian]{babel}
\usepackage{tempora}

\begin{document}

\begin{center}
    \textbf{Т"=Банк}
\end{center}

\textit{Scala} "--- язык программирования, сочетающий объектно"=ориентированное и функциональное программирование, работающий на JVM (Java Virtual Machine).

\textit{Kafka} "--- распределённая потоковая платформа для обработки событий в реальном времени, используемая для построения высоконагруженных data"=пайплайнов и event"=driven архитектур.

\textit{Kotlin} "--- cовременный статически типизированный язык программирования, работающий на JVM, Android и в браузере. Разработан JetBrains, полностью совместим с Java. 

\textit{Angulan JS} "--- фреймворк для разработки одностраничных приложений (SPA) на JavaScript. Позже заменён на Angular (версии 2+), который использует TypeScript.

\textit{Go (Golang)} "--- язык программирования от Google, отличающийся простотой, высокой производительностью и поддержкой многопоточности (горутины). Используется для бэкенда и системного программирования.

\textit{ML (Machine Learning)} "--- раздел искусственного интеллекта, изучающий алгоритмы, которые позволяют компьютерам обучаться на данных и делать предсказания без явного программирования.

\textit{React} "--- это популярная JavaScript"=библиотека с открытым исходным кодом, разработанная Facebook (Meta) для создания интерактивных пользовательских интерфейсов (UI).

\textit{Redis} "--- высокопроизводительная NoSQL база данных типа <<ключ-значение>> с поддержкой различных структур данных (строки, хеши, списки). Работает в памяти с возможностью persistence.

\textit{UX (User Experience)} "--- проектирование удобного и логичного взаимодействия пользователя с продуктом.

\textit{UI (User Interface)} "--- визуальное оформление интерфейса (кнопки, шрифты, цвета).

\textit{Docker} "--- платформа для контейнеризации приложений, позволяющая разворачивать ПО в изолированных средах (контейнерах) для упрощения разработки и деплоя.

\textit{SRE (Site Reliability Engineering)} "--- подход к управлению ИТ"=инфраструктурой, разработанный Google, сочетающий разработку и администрирование для обеспечения высокой надёжности сервисов.

\textit{Swift} - язык программирования от Apple для разработки приложений под iOS, macOS, watchOS и tvOS. Безопасный, быстрый и современный.

\end{document}